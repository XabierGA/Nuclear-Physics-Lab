\documentclass[a4paper,12pt]{article} 
\usepackage[english]{babel}
\usepackage[utf8]{inputenc}
\usepackage[T1]{fontenc} % Permite cambiar la fuente por defecto.
\usepackage{graphicx}    % Permite implementar imágenes.
\usepackage{color}       % Permite el uso de colores.
\usepackage{anysize}     % Permite modificar el tamaño de los márgenes.
\usepackage{multicol}    % Permite escribir a doble, triple...columna.
\usepackage{bm}         
\usepackage{textcomp}    
\usepackage{eurosym}     
\usepackage{amsthm} 
\usepackage{amsmath}     
\usepackage{amsmath,amsfonts} 
\usepackage{lineno} 
\usepackage{float}
\usepackage{booktabs}     
\usepackage{fancyhdr}
\usepackage{longtable}
\usepackage[makeroom]{cancel}
\usepackage{listings}

\marginsize{2.5cm}{1.5cm}{1.5cm}{1.5cm} % MÁRGENES: Izq, Der, Sup, Inf.
\parindent=0mm                        % Sangría por defecto. 
\parskip=3mm                          % Espacio entre párrafos por defecto.
\renewcommand{\baselinestretch}{1}    % Interlineado.
\newcommand{\es}{\hspace{0.15cm}}
\newcommand{\vect}[1]{\boldsymbol{#1}}
\pagestyle{fancy}
\fancyhf{}
\rhead{Xabier G. Andrade}
\rfoot{Nuclear Physics}
\lfoot{Lab}



\title{Nuclear Physics Lab}
\author{Xabier García Andrade}
\date{10th of October 2018} 

\begin{document}

\maketitle
\tableofcontents

\newpage

\section{First exercise}

\subsection{1.1}

The signal has this kind of shape because it comes from the photomultiplier and then goes into the oscilloscope and a capacitor produces it. Then the exponential decay is due to the source.

\begin{figure}[H]
\centering
\label{fig:without}
\includegraphics[width=0.8 \textwidth, inner]{withoutamplif.pdf}
\end{figure}


\subsection{1.2}


\begin{figure}[H]
\centering
\label{fig:with}
\includegraphics[width=0.8 \textwidth, inner]{amplif.pdf}
\end{figure}

\begin{figure}[H]
\centering
\label{fig:withouta}
\includegraphics[width=0.8 \textwidth, inner]{withoutamplif.pdf}
\end{figure}

The signal without amplification has an exponential shape, while after the amplification it has a gaussian shape. It is more convenient to have a gaussian shape because it has a more defined peak and the mathematical description is easier.

\subsection{1.3}

\begin{figure}[H]
\centering
\label{fig:withouta}
\includegraphics[width=0.8 \textwidth, inner]{channeldiag.pdf}
\end{figure}

\subsection{1.4}

The difference lies in the live time. Since the second one was collimated, the beam will be more focused and thus reducing the dead time.

\subsection{1.5}

Real time is the elapsed time (clock time) and live time refers to the time that the detector is available to accept pule.

\newpage

\subsection{1.6}
First one:
$$Count = \frac{13950}{59.20} = 235.64 $$	

Relative error: 

$$ \frac{235.64-235.64}{235.64} = 0$$

Second one: 
$$Count = \frac{129907}{55.54} = 2338.98 $$

Relative error: 

$$ \frac{2338.98-2338.89}{2338.89
} = 0.00004$$

\section{2 Study of gamma radiation using an HPGe detector}

\subsection{2.1}

The first one at $511.4 keV$ is electron positron annihilation due to beta plus decay and the second one at $1272.9 keV$

\subsection{2.3}

There are two main reasons behind this: 

First it´s because after positron-electron anhilation, one gamma is emitted in the opposite direction and thus it is more unlikely that it will reach the Detector. 
The second reason is because the efficiency of the detector decreases as the energy increases.

\subsection{2.4}

The Compton edge is at $331.4 kEV$. 

Relative error with the theoretical value: $$ \frac{0.334-0.331}{0.331} = 0.0091 $$

\subsection{2.5}

The first peak at $ 13308 $ refers to a beta decay to nickel 60. Then the second peak at $1130.6 keV $ to go back to the ground state.

\subsection{2.6}

The Compton Edge is at $ 976.5 keV$. 

Relative error with the theoretical value: $$ \frac{976.5-960.0}{960.0} = 0.017 $$

\subsection{2.7}

Because the other ones are not as probable.

\subsection{2.8}

Potassium-40 at $1.459 MeV$.

\subsection{2.9}

We could use a sample that would give us more peaks for the calibration.

\section{3 Radioactive Isotope Identification}

\subsection{Peaks}

$$608.9 \hspace{0.2cm} keV $$ Pb 214 gamma decay.
$$1119.0 \hspace{0.2cm} keV $$ Bi - 214 gamma decay.
$$1235.63 \hspace{0.2cm} keV $$ 
$$1458.3 \hspace{0.2cm}  keV$$ K -40 gamma decay.




\section{4 Study of alpha radiation using a surface barrier detector}

\subsection{4.1}

After looking into the table of isotopes, the values are in the order of magnitude of MeV. The best resolution would be MeV. Values for alphas: 

$$E_1 = 5.846 \hspace{0.2cm} MeV \hspace{3cm} E_2 = 5.443 \hspace{0.2cm} MeV$$

\subsection{4.2}

The maximun energy is $5.846 \hspace{0.2cm} MeV$. Our calculated value is $5.63 \hspace{0.2cm} MeV$. The difference between our values is: 

$$\frac{5.645-5.486}{5.486} = 0.02$$

\subsection{4.4}

As we pull out the sample, the energy decreases and the live time necessary increases. We get less counts per second.

\subsection{4.5}

\begin{figure}[H]
\centering
\label{fig:withouta}
\includegraphics[width=0.8 \textwidth, inner]{countdis.pdf}
\end{figure}

This decreasing exponential behaviour was expected because it follows this rule: 

$$N = N_0 e^{- \sigma x}$$

\subsection{4.6}

\begin{figure}[H]
\centering
\label{fig:withouta}
\includegraphics[width=0.8 \textwidth, inner]{enerdis.pdf}
\end{figure}

\begin{figure}[H]
\centering
\label{fig:withouta}
\includegraphics[width=0.8 \textwidth, inner]{enerdisfit.pdf}
\end{figure}

$$y = a \cdot x + b $$

$$a = (-4.09 \pm 0.20) \hspace{0.2cm} keV/mm \hspace{1cm} b = (176.0 \pm 5.4) \hspace{0.2cm} keV$$

\subsection{4.2.1}

Result: $$E = 46.98 keV$$

Thickness = $7.0 \mu m$

\section{5}

\subsection{•}


\end{document}